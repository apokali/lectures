\documentclass[a4paper,10pt]{extarticle}
\usepackage[utf8]{inputenc}
\usepackage[margin=1in]{geometry}
\usepackage{amsmath}
\usepackage{titlesec}
\titleformat*{\section}{\large\it}

\begin{document}
\begin{center}
    \large\bfseries AFM 372 \qquad Formulas for Exams \qquad Spring 2020
\end{center}
\section*{Capital Budgeting}
\begin{itemize}
    \item Capital cost of allowance
\end{itemize}

\section*{Fundamentals of Capital Budgeting}
\begin{itemize}
    \item UCC and CAA for year $t$, and \text{CCA$_t =$ UCC$_t\times d$}
    \[\text{UCC}_t=
    \begin{cases}
        \text{CapEx}/2&\text{for $t=1$} \\
        \text{CapEx}\times(1-d/2)\times(1-d)^{t-2}&\text{for $t\geq1$}
    \end{cases}\]
    
    \item Free Cash Flow
    \begin{align*}
        \text{FCF}_t=&(\text{Revenues}_t-\text{Costs}_t)\times(1-\tau_t)-\text{CapEx}_t-\Delta\text{NWC}_t\\&+\tau_c\times\text{CCA}_t
    \end{align*}
    
    \item PV of CCA tax shields
    \[\text{PV}_\text{CCA tax shields} = \dfrac{\text{CapEx}\times d \times \tau_c}{r+d}\times\left[\dfrac{1+r/2}{1+r}\right]\]
    
    \item PV of lost CCA tax shields
    \[\text{PV}_{\text{lost CCA tax shield}}=\dfrac{\min(\text{Sale Price},\text{CapEx})\times d\times \tau_c}{r+d}\times\dfrac{1}{(1+r)^t}\]
\end{itemize}

\section*{Estimating the Cost of Capital}
\begin{itemize}
    \item CAPM
    \[r_i=r_f+\beta_i\times \underbrace{(E[R_{Mkt}]-r_f)}_{\text{market risk premium}} \]
    
    \item expected return of the bond is 
    \begin{flalign*}
        r_D=(1-p)y+p(y-L)&=y-pL\\&=\textrm{Yield to Maturity $-$ Prob(default)}\times\textrm{Expected Loss Rate}
    \end{flalign*}
    
    \item the relationship between different ratios
    \[\dfrac{D}{E}=x \implies \dfrac{E}{V}=\dfrac{1}{x+1} \implies \dfrac{D}{V}=1-\dfrac{E}{V}\]
    
    \item asset or unlevered cost of capital, where $D$ is the net debt (also known as pretax WACC)
    \[r_U=\dfrac{E}{E+D}r_E+\dfrac{D}{E+D}r_D\]
    
    \item asset or unlevered beta
    \[\beta_U=\dfrac{E}{E+D}\beta_E+\dfrac{D}{E+D}\beta_E\]
    
    \item weighted average cost of capital with firm's corporate tax rate $\tau_C$
    \[r_{wacc}=\dfrac{E}{E+D}r_E+\dfrac{D}{E+D}r_D(1-\tau_C)\]
    \end{itemize}

\section*{Capital Structure of a Perfect Market}
\begin{itemize}
    \item cost of capital of levered equity
    \[r_E=r_U+\dfrac{D}{E}(r_U-r_D)\]
    
    \item In a setting of perfect capital markets, there are no taxes, so the firm's WACC and unlevered cost of capital conincide
    \[r_{wacc}=r_U=r_A\]
    
    \item a firm's unlevered or asset beta is the weighed average of its equity and debt beta
    \[\beta_U=\dfrac{E}{E+D}\beta_E+\dfrac{D}{E+D}\beta_D\]
    
    \item a firm's equity beta 
    \[\beta_E=\beta_U+\dfrac{D}{E}(\beta_U-\beta_E)\]
\end{itemize}

\clearpage

\section*{Debt and Taxes}
\begin{itemize}
    \item Interest Tax Shield in the all-equity case the total amount available for all investors is $(\mathrm{EBIT}-\mathrm{Interest})(1-\tau_c)$, but in the levered case it is
    \begin{align*}
    &\underbrace{(\mathrm{EBIT}-\mathrm{Interest})(1-\tau_c)}_{\text{Available to shareholders}}+ \underbrace{\mathrm{Interest}}_{\text{Paid to debtholders}}\\
    =&\,\underbrace{\mathrm{EBIT}(1-\tau_c)}_{\substack{\text{Cash flows to investors}\\\text{ without leverage}}}+\underbrace{\mathrm{Interest}\times\tau_c}_{\text{Interest tax shield}}
    \end{align*}
    
    \item the total value of the levered firm exceeds the value of the firm without leverage due to the PV of the tax savings from debt with corporate taxes
    \[V^L=V^U+\mathrm{PV}(\textrm{Interest tax shield})\]
    
    \item PV of the Interest tax shield given permanent debt
    \begin{align*}
        \mathrm{PV}(\text{Interst tax shield}) &= \text{PV}(\tau_c \times \text{Future interest payments})\\&=\tau_c\times\text{PV}(\text{Future interest payments})\\&=\tau_c \times D
    \end{align*}
    
    \item the weighted average cost of capital given after-tax interest rate
    \begin{align*}
        r_{wacc} &= \dfrac{E}{E+D}r_E+\dfrac{D}{E+D}r_D(1-\tau_c)\\&=\underbrace{\dfrac{E}{E+D}r_E+\dfrac{D}{E+D}r_D}_{\text{Pretax WACC}}-\underbrace{\dfrac{D}{E+D}r_D\tau_c}_{\substack{\text{Decrease from }\\ \text{interest tax shield}}}
    \end{align*}
    
    \item the effective tax advantage of debt is 
    \[\tau^{*}=1-\dfrac{(1-\tau_c)(1-\tau_e)}{(1-\tau_i)}\]
\end{itemize}
\section*{Financial Distress, Managerial Incentives, and Information}
\begin{itemize}
    \item the total value of a levered firm equals the value of the firm without leverage plus the PV of the tax savings from debt, less the PV of financial distress costs
    \[V^L=V^U+\text{PV}(\text{Interest tax shield})-\text{PV}(\text{Financial distress costs})\]
\end{itemize}

\end{document}
